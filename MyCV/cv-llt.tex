%%%%%%%%%%%%%%%
% This CV example/template is based on my own
% CV which I (lamely attempted) to clean up, so that
% it's less of an eyesore and easier for others to use.
%
% LianTze Lim (liantze@gmail.com)
% 23 Oct, 2022
%
\documentclass[a4paper,skipsamekey,11pt,english]{curve}
% Uncomment to enable Chinese; needs XeLaTeX
% \usepackage{ctex}


% Default biblatex style used for the publication list is APA6. If you wish to use a different style or pass other options to biblatex you can change them here. 
\PassOptionsToPackage{style=ieee,sorting=ydnt,uniquename=init,defernumbers=true}{biblatex}

% Most commands and style definitions are in settings.sty.
\usepackage{settings}

% If you need to further customise your biblatex setup e.g. with \DeclareFieldFormat etc please add them here AFTER loading settings.sty. For example, to remove the default "[Online] Available:" prefix before URLs when using the IEEE style:
\DefineBibliographyStrings{english}{url={\textsc{url}}}

%% Only needed if you want a Publication List
\addbibresource{own-bib.bib}

%% Specify your last name(s) and first name(s) (as given in the .bib) to automatically bold your own name in the publications list. 
%% One caveat: You need to write \bibnamedelima where there's a space in your name for this to work properly; or write \bibnamedelimi if you use initials in the .bib
% \mynames{Lim/Lian\bibnamedelima Tze}

%% You can specify multiple names like this, especially if you have changed your name or if you need to highlight multiple authors. See items 6–9 in the example "Journal Articles" output.
\mynames{Lim/Lian\bibnamedelima Tze,
  Wong/Lian\bibnamedelima Tze,
  Lim/Tracy,
  Lim/L.\bibnamedelimi T.}
%% MAKE SURE THERE IS NO SPACE AFTER THE FINAL NAME IN YOUR \mynames LIST


% Change the fonts if you want
\ifxetexorluatex % If you're using XeLaTeX or LuaLaTeX
  \usepackage{fontspec} 
  %% You can use \setmainfont etc; I'm just using these font packages here because they provide OpenType fonts for use by XeLaTeX/LuaLaTeX anyway
  \usepackage[p,osf,swashQ]{cochineal}
  \usepackage[medium,bold]{cabin}
  \usepackage[varqu,varl,scale=0.9]{zi4}
\else % If you're using pdfLaTeX or latex
  \usepackage[T1]{fontenc}
  \usepackage[p,osf,swashQ]{cochineal}
  \usepackage{cabin}
  \usepackage[varqu,varl,scale=0.9]{zi4}
\fi

% Change the page margins if you want
% \geometry{left=1cm,right=1cm,top=1.5cm,bottom=1.5cm}

% Change the colours if you want
% \definecolor{SwishLineColour}{HTML}{00FFFF}
% \definecolor{MarkerColour}{HTML}{0000CC}

% Change the item prefix marker if you want
% \prefixmarker{$\diamond$}

%% Photo is only shown if "fullonly" is included
\includecomment{fullonly}
% \excludecomment{fullonly}


%%%%%%%%%%%%%%%%%%%%%%%%%%%%%%%%%%%%%%


\leftheader{%
  {\LARGE\bfseries\sffamily Omid Sadeghnezhad}

  \makefield{\faEnvelope}{\href{mailto:omidsadeghnezhad97@gmail.com}{\texttt{omidsadeghnezhad97@gmail.com}}}
  \makefield{\faGithub}{\href{https://github.com/OmidSa75}{\texttt{OmidSa75}}}
  \makefield{\faLinkedin}
  {\href{https://www.linkedin.com/in/omid-sadeghnezhad/}{\texttt{omid-sadeghnezhad}}}

  %% Next line
  \makefield{\faGlobe}{\url{https://www.sadeghnezhad.me}}
  % \makefield{\faMapMarker}{No. 24 - 20th Tavakoli Alley - 2nd Janabz St. - Ferdowsi Blvd - Mashhad - Iran}
  \makefield{\faFlag}{Iranian}
  % \makefield{\faCalendar}{03/02/1997}

  % You can use a tabular here if you want to line up the fields.
}


\rightheader{~}
\begin{fullonly}
% \photo[r]{photo}
% \photoscale{0.13}
\end{fullonly}

% \title{Curriculum Vitae}

\begin{document}
\makeheaders[c]

% \makerubric{employment}

\makerubrichead{\makefield{\faInfoCircle}{Profile}}


\justifying
\normalfont Experienced Machine Learning Engineer and Python Developer specializing in AI and Computer Vision for more than 5 years. Skilled in high-performance programming, ML model implementation, and system optimization. Proficient in leading technical teams and integrating cutting-edge AI technologies. Seeking opportunities to apply expertise in AI and computer vision to drive innovation and solve complex problems.

\makerubrichead{\makefield{\faPuzzlePiece}{Research Interests}}

% \begin{rubric}{\makefield{\faPuzzlePiece}{Research Interests}}
% \rubricspace{1pt}
    \itemfeature{Generative Adversarial Networks and Diffusion models especially in computer vision and image processing fields}
    \itemfeature{Object Detection and Tracking models}
    \itemfeature{Image processing and computer vision} 
    \itemfeature{Signal processing and analyzing}
% \end{rubric}


\makerubric{education}
% \makerubrichead{\makefield{\faBook}{Courses}}



\begin{tabular}{p{0.3\textwidth}p{0.3\textwidth}p{0.3\textwidth}}
{\textbullet}{Neural Networks and Deep Learning}  & {\textbullet}{Machine Learning}  & {\textbullet}{Advanced Data Mining} \\
{\textbullet}{Advanced Information Retrieval}  & {\textbullet}{Statistical Pattern Recognition}  & {\textbullet}{Introduction to Machine Learning in Production} \\
\end{tabular}
% \addfeature{\textbullet}{Neural Networks and Deep Learning}  \addfeature{\textbullet}{Machine Learning}

\makerubric{professional_experience}
% If you're not a researcher nor an academic, you probably don't have any publications; delete this line.
%% Sometimes when a section can't be nicely modelled with the \entry[]... mechanism; hack our own and use \input NOT \makerubric
% %% Sometimes when a section can't be nicely modelled with the \entry[]... mechanism; hack our own
\makerubrichead{Research Publications}

%% Assuming you've already given \addbibresource{own-bib.bib} in the main doc. Right? Right???
\nocite{*}

%% If you just want everything in one list
% \printbibliography[heading={none}]

\printbibliography[heading={subbibliography},title={Journal Articles},type=article]

\printbibliography[heading={subbibliography},title={Conference Proceedings},type=inproceedings]

\printbibliography[heading={subbibliography},title={Books and Chapters},filter={booksandchapters}]



\makerubric{skills}
\makerubrichead{\makefield{\faBook}{Research Experience}}

\itemfeature{Retrieval-augmented generation - \textit{2024}}
\itemfeature{ Evolutionary Algorithms and implementing Symbiotic Organisms Search algorithm (SOS) - \textit{2023}}
\itemfeature{GAN Developments survey and Analyze the Latent space - \textit{2022}}
\itemfeature{Clothes virtual try-on models survey - \textit{2020}}
\itemfeature{Image harmonization and blending methods, color transfer methods, and color constancy with image to image translation - \textit{2020}}
\itemfeature{Image depth extractions and salient object detection models - \textit{2020}}



% \makerubric{misc}
\makerubric{projects}
\makerubrichead{\makefield{\faGamepad}{Interests}}



\begin{tabular}{p{0.3\textwidth}p{0.3\textwidth}p{0.3\textwidth}}
\makefield{\textbullet}{Guitar}  & \makefield{\textbullet}{Video Game}  & \makefield{\textbullet}{Ping Pong} \\
\makefield{\textbullet}{Nature Lover}  & \makefield{\textbullet}{Psychology \& Philosophy}   \\
\end{tabular}

% \makerubric{referee}

% %% Probably not the best way of doing it but what the heck, I just winged-it :p

\makerubrichead{References}

\begin{tabularx}{\textwidth}{@{}X X@{}}
\textbf{Prof X}\par
Professor\par
ABC University,\par 
Address.\par 
\makefield{\faEnvelopeO}{\url{abc@def.edu}}
& 
\textbf{Prof Y}\par
Professor\par
ABC University,\par 
Address.\par 
\makefield{\faEnvelopeO}{\url{abc@def.edu}}
\\
\end{tabularx}


\end{document}